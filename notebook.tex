
% Default to the notebook output style

    


% Inherit from the specified cell style.




    
\documentclass[11pt]{article}

    
    
    \usepackage[T1]{fontenc}
    % Nicer default font (+ math font) than Computer Modern for most use cases
    \usepackage{mathpazo}

    % Basic figure setup, for now with no caption control since it's done
    % automatically by Pandoc (which extracts ![](path) syntax from Markdown).
    \usepackage{graphicx}
    % We will generate all images so they have a width \maxwidth. This means
    % that they will get their normal width if they fit onto the page, but
    % are scaled down if they would overflow the margins.
    \makeatletter
    \def\maxwidth{\ifdim\Gin@nat@width>\linewidth\linewidth
    \else\Gin@nat@width\fi}
    \makeatother
    \let\Oldincludegraphics\includegraphics
    % Set max figure width to be 80% of text width, for now hardcoded.
    \renewcommand{\includegraphics}[1]{\Oldincludegraphics[width=.8\maxwidth]{#1}}
    % Ensure that by default, figures have no caption (until we provide a
    % proper Figure object with a Caption API and a way to capture that
    % in the conversion process - todo).
    \usepackage{caption}
    \DeclareCaptionLabelFormat{nolabel}{}
    \captionsetup{labelformat=nolabel}

    \usepackage{adjustbox} % Used to constrain images to a maximum size 
    \usepackage{xcolor} % Allow colors to be defined
    \usepackage{enumerate} % Needed for markdown enumerations to work
    \usepackage{geometry} % Used to adjust the document margins
    \usepackage{amsmath} % Equations
    \usepackage{amssymb} % Equations
    \usepackage{textcomp} % defines textquotesingle
    % Hack from http://tex.stackexchange.com/a/47451/13684:
    \AtBeginDocument{%
        \def\PYZsq{\textquotesingle}% Upright quotes in Pygmentized code
    }
    \usepackage{upquote} % Upright quotes for verbatim code
    \usepackage{eurosym} % defines \euro
    \usepackage[mathletters]{ucs} % Extended unicode (utf-8) support
    \usepackage[utf8x]{inputenc} % Allow utf-8 characters in the tex document
    \usepackage{fancyvrb} % verbatim replacement that allows latex
    \usepackage{grffile} % extends the file name processing of package graphics 
                         % to support a larger range 
    % The hyperref package gives us a pdf with properly built
    % internal navigation ('pdf bookmarks' for the table of contents,
    % internal cross-reference links, web links for URLs, etc.)
    \usepackage{hyperref}
    \usepackage{longtable} % longtable support required by pandoc >1.10
    \usepackage{booktabs}  % table support for pandoc > 1.12.2
    \usepackage[inline]{enumitem} % IRkernel/repr support (it uses the enumerate* environment)
    \usepackage[normalem]{ulem} % ulem is needed to support strikethroughs (\sout)
                                % normalem makes italics be italics, not underlines
    

    
    
    % Colors for the hyperref package
    \definecolor{urlcolor}{rgb}{0,.145,.698}
    \definecolor{linkcolor}{rgb}{.71,0.21,0.01}
    \definecolor{citecolor}{rgb}{.12,.54,.11}

    % ANSI colors
    \definecolor{ansi-black}{HTML}{3E424D}
    \definecolor{ansi-black-intense}{HTML}{282C36}
    \definecolor{ansi-red}{HTML}{E75C58}
    \definecolor{ansi-red-intense}{HTML}{B22B31}
    \definecolor{ansi-green}{HTML}{00A250}
    \definecolor{ansi-green-intense}{HTML}{007427}
    \definecolor{ansi-yellow}{HTML}{DDB62B}
    \definecolor{ansi-yellow-intense}{HTML}{B27D12}
    \definecolor{ansi-blue}{HTML}{208FFB}
    \definecolor{ansi-blue-intense}{HTML}{0065CA}
    \definecolor{ansi-magenta}{HTML}{D160C4}
    \definecolor{ansi-magenta-intense}{HTML}{A03196}
    \definecolor{ansi-cyan}{HTML}{60C6C8}
    \definecolor{ansi-cyan-intense}{HTML}{258F8F}
    \definecolor{ansi-white}{HTML}{C5C1B4}
    \definecolor{ansi-white-intense}{HTML}{A1A6B2}

    % commands and environments needed by pandoc snippets
    % extracted from the output of `pandoc -s`
    \providecommand{\tightlist}{%
      \setlength{\itemsep}{0pt}\setlength{\parskip}{0pt}}
    \DefineVerbatimEnvironment{Highlighting}{Verbatim}{commandchars=\\\{\}}
    % Add ',fontsize=\small' for more characters per line
    \newenvironment{Shaded}{}{}
    \newcommand{\KeywordTok}[1]{\textcolor[rgb]{0.00,0.44,0.13}{\textbf{{#1}}}}
    \newcommand{\DataTypeTok}[1]{\textcolor[rgb]{0.56,0.13,0.00}{{#1}}}
    \newcommand{\DecValTok}[1]{\textcolor[rgb]{0.25,0.63,0.44}{{#1}}}
    \newcommand{\BaseNTok}[1]{\textcolor[rgb]{0.25,0.63,0.44}{{#1}}}
    \newcommand{\FloatTok}[1]{\textcolor[rgb]{0.25,0.63,0.44}{{#1}}}
    \newcommand{\CharTok}[1]{\textcolor[rgb]{0.25,0.44,0.63}{{#1}}}
    \newcommand{\StringTok}[1]{\textcolor[rgb]{0.25,0.44,0.63}{{#1}}}
    \newcommand{\CommentTok}[1]{\textcolor[rgb]{0.38,0.63,0.69}{\textit{{#1}}}}
    \newcommand{\OtherTok}[1]{\textcolor[rgb]{0.00,0.44,0.13}{{#1}}}
    \newcommand{\AlertTok}[1]{\textcolor[rgb]{1.00,0.00,0.00}{\textbf{{#1}}}}
    \newcommand{\FunctionTok}[1]{\textcolor[rgb]{0.02,0.16,0.49}{{#1}}}
    \newcommand{\RegionMarkerTok}[1]{{#1}}
    \newcommand{\ErrorTok}[1]{\textcolor[rgb]{1.00,0.00,0.00}{\textbf{{#1}}}}
    \newcommand{\NormalTok}[1]{{#1}}
    
    % Additional commands for more recent versions of Pandoc
    \newcommand{\ConstantTok}[1]{\textcolor[rgb]{0.53,0.00,0.00}{{#1}}}
    \newcommand{\SpecialCharTok}[1]{\textcolor[rgb]{0.25,0.44,0.63}{{#1}}}
    \newcommand{\VerbatimStringTok}[1]{\textcolor[rgb]{0.25,0.44,0.63}{{#1}}}
    \newcommand{\SpecialStringTok}[1]{\textcolor[rgb]{0.73,0.40,0.53}{{#1}}}
    \newcommand{\ImportTok}[1]{{#1}}
    \newcommand{\DocumentationTok}[1]{\textcolor[rgb]{0.73,0.13,0.13}{\textit{{#1}}}}
    \newcommand{\AnnotationTok}[1]{\textcolor[rgb]{0.38,0.63,0.69}{\textbf{\textit{{#1}}}}}
    \newcommand{\CommentVarTok}[1]{\textcolor[rgb]{0.38,0.63,0.69}{\textbf{\textit{{#1}}}}}
    \newcommand{\VariableTok}[1]{\textcolor[rgb]{0.10,0.09,0.49}{{#1}}}
    \newcommand{\ControlFlowTok}[1]{\textcolor[rgb]{0.00,0.44,0.13}{\textbf{{#1}}}}
    \newcommand{\OperatorTok}[1]{\textcolor[rgb]{0.40,0.40,0.40}{{#1}}}
    \newcommand{\BuiltInTok}[1]{{#1}}
    \newcommand{\ExtensionTok}[1]{{#1}}
    \newcommand{\PreprocessorTok}[1]{\textcolor[rgb]{0.74,0.48,0.00}{{#1}}}
    \newcommand{\AttributeTok}[1]{\textcolor[rgb]{0.49,0.56,0.16}{{#1}}}
    \newcommand{\InformationTok}[1]{\textcolor[rgb]{0.38,0.63,0.69}{\textbf{\textit{{#1}}}}}
    \newcommand{\WarningTok}[1]{\textcolor[rgb]{0.38,0.63,0.69}{\textbf{\textit{{#1}}}}}
    
    
    % Define a nice break command that doesn't care if a line doesn't already
    % exist.
    \def\br{\hspace*{\fill} \\* }
    % Math Jax compatability definitions
    \def\gt{>}
    \def\lt{<}
    % Document parameters
    \title{NDFDSI - Projeto de An?lise de Dados do Titanic}
    
    
    

    % Pygments definitions
    
\makeatletter
\def\PY@reset{\let\PY@it=\relax \let\PY@bf=\relax%
    \let\PY@ul=\relax \let\PY@tc=\relax%
    \let\PY@bc=\relax \let\PY@ff=\relax}
\def\PY@tok#1{\csname PY@tok@#1\endcsname}
\def\PY@toks#1+{\ifx\relax#1\empty\else%
    \PY@tok{#1}\expandafter\PY@toks\fi}
\def\PY@do#1{\PY@bc{\PY@tc{\PY@ul{%
    \PY@it{\PY@bf{\PY@ff{#1}}}}}}}
\def\PY#1#2{\PY@reset\PY@toks#1+\relax+\PY@do{#2}}

\expandafter\def\csname PY@tok@w\endcsname{\def\PY@tc##1{\textcolor[rgb]{0.73,0.73,0.73}{##1}}}
\expandafter\def\csname PY@tok@c\endcsname{\let\PY@it=\textit\def\PY@tc##1{\textcolor[rgb]{0.25,0.50,0.50}{##1}}}
\expandafter\def\csname PY@tok@cp\endcsname{\def\PY@tc##1{\textcolor[rgb]{0.74,0.48,0.00}{##1}}}
\expandafter\def\csname PY@tok@k\endcsname{\let\PY@bf=\textbf\def\PY@tc##1{\textcolor[rgb]{0.00,0.50,0.00}{##1}}}
\expandafter\def\csname PY@tok@kp\endcsname{\def\PY@tc##1{\textcolor[rgb]{0.00,0.50,0.00}{##1}}}
\expandafter\def\csname PY@tok@kt\endcsname{\def\PY@tc##1{\textcolor[rgb]{0.69,0.00,0.25}{##1}}}
\expandafter\def\csname PY@tok@o\endcsname{\def\PY@tc##1{\textcolor[rgb]{0.40,0.40,0.40}{##1}}}
\expandafter\def\csname PY@tok@ow\endcsname{\let\PY@bf=\textbf\def\PY@tc##1{\textcolor[rgb]{0.67,0.13,1.00}{##1}}}
\expandafter\def\csname PY@tok@nb\endcsname{\def\PY@tc##1{\textcolor[rgb]{0.00,0.50,0.00}{##1}}}
\expandafter\def\csname PY@tok@nf\endcsname{\def\PY@tc##1{\textcolor[rgb]{0.00,0.00,1.00}{##1}}}
\expandafter\def\csname PY@tok@nc\endcsname{\let\PY@bf=\textbf\def\PY@tc##1{\textcolor[rgb]{0.00,0.00,1.00}{##1}}}
\expandafter\def\csname PY@tok@nn\endcsname{\let\PY@bf=\textbf\def\PY@tc##1{\textcolor[rgb]{0.00,0.00,1.00}{##1}}}
\expandafter\def\csname PY@tok@ne\endcsname{\let\PY@bf=\textbf\def\PY@tc##1{\textcolor[rgb]{0.82,0.25,0.23}{##1}}}
\expandafter\def\csname PY@tok@nv\endcsname{\def\PY@tc##1{\textcolor[rgb]{0.10,0.09,0.49}{##1}}}
\expandafter\def\csname PY@tok@no\endcsname{\def\PY@tc##1{\textcolor[rgb]{0.53,0.00,0.00}{##1}}}
\expandafter\def\csname PY@tok@nl\endcsname{\def\PY@tc##1{\textcolor[rgb]{0.63,0.63,0.00}{##1}}}
\expandafter\def\csname PY@tok@ni\endcsname{\let\PY@bf=\textbf\def\PY@tc##1{\textcolor[rgb]{0.60,0.60,0.60}{##1}}}
\expandafter\def\csname PY@tok@na\endcsname{\def\PY@tc##1{\textcolor[rgb]{0.49,0.56,0.16}{##1}}}
\expandafter\def\csname PY@tok@nt\endcsname{\let\PY@bf=\textbf\def\PY@tc##1{\textcolor[rgb]{0.00,0.50,0.00}{##1}}}
\expandafter\def\csname PY@tok@nd\endcsname{\def\PY@tc##1{\textcolor[rgb]{0.67,0.13,1.00}{##1}}}
\expandafter\def\csname PY@tok@s\endcsname{\def\PY@tc##1{\textcolor[rgb]{0.73,0.13,0.13}{##1}}}
\expandafter\def\csname PY@tok@sd\endcsname{\let\PY@it=\textit\def\PY@tc##1{\textcolor[rgb]{0.73,0.13,0.13}{##1}}}
\expandafter\def\csname PY@tok@si\endcsname{\let\PY@bf=\textbf\def\PY@tc##1{\textcolor[rgb]{0.73,0.40,0.53}{##1}}}
\expandafter\def\csname PY@tok@se\endcsname{\let\PY@bf=\textbf\def\PY@tc##1{\textcolor[rgb]{0.73,0.40,0.13}{##1}}}
\expandafter\def\csname PY@tok@sr\endcsname{\def\PY@tc##1{\textcolor[rgb]{0.73,0.40,0.53}{##1}}}
\expandafter\def\csname PY@tok@ss\endcsname{\def\PY@tc##1{\textcolor[rgb]{0.10,0.09,0.49}{##1}}}
\expandafter\def\csname PY@tok@sx\endcsname{\def\PY@tc##1{\textcolor[rgb]{0.00,0.50,0.00}{##1}}}
\expandafter\def\csname PY@tok@m\endcsname{\def\PY@tc##1{\textcolor[rgb]{0.40,0.40,0.40}{##1}}}
\expandafter\def\csname PY@tok@gh\endcsname{\let\PY@bf=\textbf\def\PY@tc##1{\textcolor[rgb]{0.00,0.00,0.50}{##1}}}
\expandafter\def\csname PY@tok@gu\endcsname{\let\PY@bf=\textbf\def\PY@tc##1{\textcolor[rgb]{0.50,0.00,0.50}{##1}}}
\expandafter\def\csname PY@tok@gd\endcsname{\def\PY@tc##1{\textcolor[rgb]{0.63,0.00,0.00}{##1}}}
\expandafter\def\csname PY@tok@gi\endcsname{\def\PY@tc##1{\textcolor[rgb]{0.00,0.63,0.00}{##1}}}
\expandafter\def\csname PY@tok@gr\endcsname{\def\PY@tc##1{\textcolor[rgb]{1.00,0.00,0.00}{##1}}}
\expandafter\def\csname PY@tok@ge\endcsname{\let\PY@it=\textit}
\expandafter\def\csname PY@tok@gs\endcsname{\let\PY@bf=\textbf}
\expandafter\def\csname PY@tok@gp\endcsname{\let\PY@bf=\textbf\def\PY@tc##1{\textcolor[rgb]{0.00,0.00,0.50}{##1}}}
\expandafter\def\csname PY@tok@go\endcsname{\def\PY@tc##1{\textcolor[rgb]{0.53,0.53,0.53}{##1}}}
\expandafter\def\csname PY@tok@gt\endcsname{\def\PY@tc##1{\textcolor[rgb]{0.00,0.27,0.87}{##1}}}
\expandafter\def\csname PY@tok@err\endcsname{\def\PY@bc##1{\setlength{\fboxsep}{0pt}\fcolorbox[rgb]{1.00,0.00,0.00}{1,1,1}{\strut ##1}}}
\expandafter\def\csname PY@tok@kc\endcsname{\let\PY@bf=\textbf\def\PY@tc##1{\textcolor[rgb]{0.00,0.50,0.00}{##1}}}
\expandafter\def\csname PY@tok@kd\endcsname{\let\PY@bf=\textbf\def\PY@tc##1{\textcolor[rgb]{0.00,0.50,0.00}{##1}}}
\expandafter\def\csname PY@tok@kn\endcsname{\let\PY@bf=\textbf\def\PY@tc##1{\textcolor[rgb]{0.00,0.50,0.00}{##1}}}
\expandafter\def\csname PY@tok@kr\endcsname{\let\PY@bf=\textbf\def\PY@tc##1{\textcolor[rgb]{0.00,0.50,0.00}{##1}}}
\expandafter\def\csname PY@tok@bp\endcsname{\def\PY@tc##1{\textcolor[rgb]{0.00,0.50,0.00}{##1}}}
\expandafter\def\csname PY@tok@fm\endcsname{\def\PY@tc##1{\textcolor[rgb]{0.00,0.00,1.00}{##1}}}
\expandafter\def\csname PY@tok@vc\endcsname{\def\PY@tc##1{\textcolor[rgb]{0.10,0.09,0.49}{##1}}}
\expandafter\def\csname PY@tok@vg\endcsname{\def\PY@tc##1{\textcolor[rgb]{0.10,0.09,0.49}{##1}}}
\expandafter\def\csname PY@tok@vi\endcsname{\def\PY@tc##1{\textcolor[rgb]{0.10,0.09,0.49}{##1}}}
\expandafter\def\csname PY@tok@vm\endcsname{\def\PY@tc##1{\textcolor[rgb]{0.10,0.09,0.49}{##1}}}
\expandafter\def\csname PY@tok@sa\endcsname{\def\PY@tc##1{\textcolor[rgb]{0.73,0.13,0.13}{##1}}}
\expandafter\def\csname PY@tok@sb\endcsname{\def\PY@tc##1{\textcolor[rgb]{0.73,0.13,0.13}{##1}}}
\expandafter\def\csname PY@tok@sc\endcsname{\def\PY@tc##1{\textcolor[rgb]{0.73,0.13,0.13}{##1}}}
\expandafter\def\csname PY@tok@dl\endcsname{\def\PY@tc##1{\textcolor[rgb]{0.73,0.13,0.13}{##1}}}
\expandafter\def\csname PY@tok@s2\endcsname{\def\PY@tc##1{\textcolor[rgb]{0.73,0.13,0.13}{##1}}}
\expandafter\def\csname PY@tok@sh\endcsname{\def\PY@tc##1{\textcolor[rgb]{0.73,0.13,0.13}{##1}}}
\expandafter\def\csname PY@tok@s1\endcsname{\def\PY@tc##1{\textcolor[rgb]{0.73,0.13,0.13}{##1}}}
\expandafter\def\csname PY@tok@mb\endcsname{\def\PY@tc##1{\textcolor[rgb]{0.40,0.40,0.40}{##1}}}
\expandafter\def\csname PY@tok@mf\endcsname{\def\PY@tc##1{\textcolor[rgb]{0.40,0.40,0.40}{##1}}}
\expandafter\def\csname PY@tok@mh\endcsname{\def\PY@tc##1{\textcolor[rgb]{0.40,0.40,0.40}{##1}}}
\expandafter\def\csname PY@tok@mi\endcsname{\def\PY@tc##1{\textcolor[rgb]{0.40,0.40,0.40}{##1}}}
\expandafter\def\csname PY@tok@il\endcsname{\def\PY@tc##1{\textcolor[rgb]{0.40,0.40,0.40}{##1}}}
\expandafter\def\csname PY@tok@mo\endcsname{\def\PY@tc##1{\textcolor[rgb]{0.40,0.40,0.40}{##1}}}
\expandafter\def\csname PY@tok@ch\endcsname{\let\PY@it=\textit\def\PY@tc##1{\textcolor[rgb]{0.25,0.50,0.50}{##1}}}
\expandafter\def\csname PY@tok@cm\endcsname{\let\PY@it=\textit\def\PY@tc##1{\textcolor[rgb]{0.25,0.50,0.50}{##1}}}
\expandafter\def\csname PY@tok@cpf\endcsname{\let\PY@it=\textit\def\PY@tc##1{\textcolor[rgb]{0.25,0.50,0.50}{##1}}}
\expandafter\def\csname PY@tok@c1\endcsname{\let\PY@it=\textit\def\PY@tc##1{\textcolor[rgb]{0.25,0.50,0.50}{##1}}}
\expandafter\def\csname PY@tok@cs\endcsname{\let\PY@it=\textit\def\PY@tc##1{\textcolor[rgb]{0.25,0.50,0.50}{##1}}}

\def\PYZbs{\char`\\}
\def\PYZus{\char`\_}
\def\PYZob{\char`\{}
\def\PYZcb{\char`\}}
\def\PYZca{\char`\^}
\def\PYZam{\char`\&}
\def\PYZlt{\char`\<}
\def\PYZgt{\char`\>}
\def\PYZsh{\char`\#}
\def\PYZpc{\char`\%}
\def\PYZdl{\char`\$}
\def\PYZhy{\char`\-}
\def\PYZsq{\char`\'}
\def\PYZdq{\char`\"}
\def\PYZti{\char`\~}
% for compatibility with earlier versions
\def\PYZat{@}
\def\PYZlb{[}
\def\PYZrb{]}
\makeatother


    % Exact colors from NB
    \definecolor{incolor}{rgb}{0.0, 0.0, 0.5}
    \definecolor{outcolor}{rgb}{0.545, 0.0, 0.0}



    
    % Prevent overflowing lines due to hard-to-break entities
    \sloppy 
    % Setup hyperref package
    \hypersetup{
      breaklinks=true,  % so long urls are correctly broken across lines
      colorlinks=true,
      urlcolor=urlcolor,
      linkcolor=linkcolor,
      citecolor=citecolor,
      }
    % Slightly bigger margins than the latex defaults
    
    \geometry{verbose,tmargin=1in,bmargin=1in,lmargin=1in,rmargin=1in}
    
    

    \begin{document}
    
    
    \maketitle
    
    

    
    \hypertarget{analisando-os-dados-dos-passageiros-do-titanic}{%
\section{Analisando os Dados dos Passageiros do
Titanic}\label{analisando-os-dados-dos-passageiros-do-titanic}}

    O RMS Titanic foi um navio de passageiros britânico operado pela White
Star Line e construído pelos estaleiros da Harland and Wolff em Belfast.
Foi a segunda embarcação da Classe Olympic de transatlânticos depois do
RMS Olympic e seguido pelo HMHS Britannic. O Titanic foi pensado para
ser o navio mais luxuoso e mais seguro de sua época, gerando lendas que
era supostamente ``inafundável''. Ele colidiu com um iceberg às 23h40min
do dia 14 de abril e afundou na madrugada do dia seguinte com mais de
1500 pessoas a bordo, sendo um dos maiores desastres marítimos em tempos
de paz de toda a história. Seu naufrágio destacou vários pontos fracos
de seu projeto, deficiências nos procedimentos de evacuação de
emergência e falhas nas regulamentações marítimas da época.

fonte:\href{https://pt.wikipedia.org/wiki/RMS_Titanic}{wikipedia}

Este documento visa analisar um conjunto de dados referente aos
passageiros do navio Titanic, os dados utilizados são apenas uma
amostra, não correspondem a totalidade de passageiros.

Irei mostrar todas as etapas necessárias para a análise. *
Questionamento. * Limpeza de dados. * Desenvolvimento das questões. *
Conclusão.

\hypertarget{base-de-dados}{%
\subsubsection{Base de dados}\label{base-de-dados}}

Esta base de dados foram obtidas do site
\href{https://www.kaggle.com/c/titanic/data}{Kaggle}

    \hypertarget{questionamento}{%
\subsection{1 - Questionamento}\label{questionamento}}

    Este projeto tem como foco obter conclusões que ajudem a sanar as
questões que serão listadas a seguir: * Qual a média de idade dos
passageiros? * Qual a média de idade por classe social e genêro dos
sobreviventes? * Qual é a classe econômica predominante? * Portos de
embarque por classe social? * Entre os sobreviventes, qual o sexo
predominou?

    \hypertarget{limpeza-dos-dados}{%
\subsection{2 - Limpeza dos Dados}\label{limpeza-dos-dados}}

    \begin{Verbatim}[commandchars=\\\{\}]
{\color{incolor}In [{\color{incolor}1}]:} \PY{c+c1}{\PYZsh{} importanto as bibliotecas necessárias.}
        \PY{k+kn}{import} \PY{n+nn}{pandas} \PY{k}{as} \PY{n+nn}{pd}
        \PY{k+kn}{import} \PY{n+nn}{numpy} \PY{k}{as} \PY{n+nn}{np}
        \PY{k+kn}{import} \PY{n+nn}{matplotlib}\PY{n+nn}{.}\PY{n+nn}{pyplot} \PY{k}{as} \PY{n+nn}{plt}
        \PY{k+kn}{import} \PY{n+nn}{seaborn} \PY{k}{as} \PY{n+nn}{sns}
        \PY{o}{\PYZpc{}} \PY{n}{matplotlib} \PY{n}{inline}
        
        \PY{c+c1}{\PYZsh{} atribuindo a leitura do arquivo a uma variavél.}
        \PY{n}{df} \PY{o}{=} \PY{n}{pd}\PY{o}{.}\PY{n}{read\PYZus{}csv}\PY{p}{(}\PY{l+s+s1}{\PYZsq{}}\PY{l+s+s1}{titanic\PYZhy{}data\PYZhy{}6.csv}\PY{l+s+s1}{\PYZsq{}}\PY{p}{)}
\end{Verbatim}


    \begin{Verbatim}[commandchars=\\\{\}]
{\color{incolor}In [{\color{incolor}2}]:} \PY{c+c1}{\PYZsh{} visualizando as informações do data frame.}
        \PY{n}{df}\PY{o}{.}\PY{n}{info}\PY{p}{(}\PY{p}{)}
\end{Verbatim}


    \begin{Verbatim}[commandchars=\\\{\}]
<class 'pandas.core.frame.DataFrame'>
RangeIndex: 891 entries, 0 to 890
Data columns (total 12 columns):
PassengerId    891 non-null int64
Survived       891 non-null int64
Pclass         891 non-null int64
Name           891 non-null object
Sex            891 non-null object
Age            714 non-null float64
SibSp          891 non-null int64
Parch          891 non-null int64
Ticket         891 non-null object
Fare           891 non-null float64
Cabin          204 non-null object
Embarked       889 non-null object
dtypes: float64(2), int64(5), object(5)
memory usage: 83.6+ KB

    \end{Verbatim}

    \begin{Verbatim}[commandchars=\\\{\}]
{\color{incolor}In [{\color{incolor}3}]:} \PY{c+c1}{\PYZsh{} Visualizando as 5 primeiras linhas do data frame.}
        \PY{n}{df}\PY{o}{.}\PY{n}{head}\PY{p}{(}\PY{p}{)}
\end{Verbatim}


\begin{Verbatim}[commandchars=\\\{\}]
{\color{outcolor}Out[{\color{outcolor}3}]:}    PassengerId  Survived  Pclass  \textbackslash{}
        0            1         0       3   
        1            2         1       1   
        2            3         1       3   
        3            4         1       1   
        4            5         0       3   
        
                                                        Name     Sex   Age  SibSp  \textbackslash{}
        0                            Braund, Mr. Owen Harris    male  22.0      1   
        1  Cumings, Mrs. John Bradley (Florence Briggs Th{\ldots}  female  38.0      1   
        2                             Heikkinen, Miss. Laina  female  26.0      0   
        3       Futrelle, Mrs. Jacques Heath (Lily May Peel)  female  35.0      1   
        4                           Allen, Mr. William Henry    male  35.0      0   
        
           Parch            Ticket     Fare Cabin Embarked  
        0      0         A/5 21171   7.2500   NaN        S  
        1      0          PC 17599  71.2833   C85        C  
        2      0  STON/O2. 3101282   7.9250   NaN        S  
        3      0            113803  53.1000  C123        S  
        4      0            373450   8.0500   NaN        S  
\end{Verbatim}
            
    \begin{Verbatim}[commandchars=\\\{\}]
{\color{incolor}In [{\color{incolor}4}]:} \PY{c+c1}{\PYZsh{} excluindo as colunas que não geram conclusões}
        \PY{n}{df}\PY{o}{.}\PY{n}{drop}\PY{p}{(}\PY{p}{[}\PY{l+s+s1}{\PYZsq{}}\PY{l+s+s1}{PassengerId}\PY{l+s+s1}{\PYZsq{}}\PY{p}{,}\PY{l+s+s1}{\PYZsq{}}\PY{l+s+s1}{Name}\PY{l+s+s1}{\PYZsq{}}\PY{p}{,}\PY{l+s+s1}{\PYZsq{}}\PY{l+s+s1}{Ticket}\PY{l+s+s1}{\PYZsq{}}\PY{p}{,}\PY{l+s+s1}{\PYZsq{}}\PY{l+s+s1}{Cabin}\PY{l+s+s1}{\PYZsq{}}\PY{p}{]}\PY{p}{,}\PY{n}{axis}\PY{o}{=}\PY{l+m+mi}{1}\PY{p}{,}\PY{n}{inplace}\PY{o}{=}\PY{k+kc}{True}\PY{p}{)}
\end{Verbatim}


    Aqui estou alterando as propriedade das tarifas para inteiro, usei o
método de conversão pelo numpy porque o método `astype' do pandas
converte em int32, é uma questão de uniformidade somente.

    \begin{Verbatim}[commandchars=\\\{\}]
{\color{incolor}In [{\color{incolor}5}]:} \PY{n}{df}\PY{p}{[}\PY{l+s+s1}{\PYZsq{}}\PY{l+s+s1}{Fare}\PY{l+s+s1}{\PYZsq{}}\PY{p}{]} \PY{o}{=} \PY{n}{np}\PY{o}{.}\PY{n}{int64}\PY{p}{(}\PY{n}{df}\PY{p}{[}\PY{l+s+s1}{\PYZsq{}}\PY{l+s+s1}{Fare}\PY{l+s+s1}{\PYZsq{}}\PY{p}{]}\PY{p}{)}
\end{Verbatim}


    \begin{Verbatim}[commandchars=\\\{\}]
{\color{incolor}In [{\color{incolor}6}]:} \PY{c+c1}{\PYZsh{} visualizando para confirmar as mudanças e exclusões.}
        \PY{n}{df}\PY{o}{.}\PY{n}{info}\PY{p}{(}\PY{p}{)}
\end{Verbatim}


    \begin{Verbatim}[commandchars=\\\{\}]
<class 'pandas.core.frame.DataFrame'>
RangeIndex: 891 entries, 0 to 890
Data columns (total 8 columns):
Survived    891 non-null int64
Pclass      891 non-null int64
Sex         891 non-null object
Age         714 non-null float64
SibSp       891 non-null int64
Parch       891 non-null int64
Fare        891 non-null int64
Embarked    889 non-null object
dtypes: float64(1), int64(5), object(2)
memory usage: 55.8+ KB

    \end{Verbatim}

    Neste ponto estou criando e preparando um variável para ser usada quando
solicitar análises com a coluna \textbf{``Age''}, pois nesta coluna
falta bastante item e não quero usar \textbf{dropna} em toda a base de
dados, isso excluiria muitas linha e poderia comprometer a análise.

    \begin{Verbatim}[commandchars=\\\{\}]
{\color{incolor}In [{\color{incolor}7}]:} \PY{c+c1}{\PYZsh{} criando um cópia do data frama para tratar os dados de \PYZsq{}Age\PYZsq{}.}
        \PY{n}{df\PYZus{}age} \PY{o}{=} \PY{n}{df}\PY{o}{.}\PY{n}{copy}\PY{p}{(}\PY{p}{)}
        
        \PY{c+c1}{\PYZsh{} apagando a coluna \PYZsq{}Embarked\PYZsq{}.}
        \PY{n}{df\PYZus{}age}\PY{o}{.}\PY{n}{drop}\PY{p}{(}\PY{p}{[}\PY{l+s+s1}{\PYZsq{}}\PY{l+s+s1}{Embarked}\PY{l+s+s1}{\PYZsq{}}\PY{p}{]}\PY{p}{,}\PY{n}{axis}\PY{o}{=}\PY{l+m+mi}{1}\PY{p}{,}\PY{n}{inplace}\PY{o}{=}\PY{k+kc}{True}\PY{p}{)}
        
        \PY{c+c1}{\PYZsh{} apagando as linhas com elementos nulos.}
        \PY{n}{df\PYZus{}age}\PY{o}{.}\PY{n}{dropna}\PY{p}{(}\PY{n}{inplace}\PY{o}{=}\PY{k+kc}{True}\PY{p}{)}
        
        \PY{c+c1}{\PYZsh{} trocando a propriedade de \PYZsq{}Age\PYZsq{} para inteiro.}
        \PY{n}{df\PYZus{}age}\PY{p}{[}\PY{l+s+s1}{\PYZsq{}}\PY{l+s+s1}{Age}\PY{l+s+s1}{\PYZsq{}}\PY{p}{]} \PY{o}{=} \PY{n}{np}\PY{o}{.}\PY{n}{int64}\PY{p}{(}\PY{n}{df\PYZus{}age}\PY{p}{[}\PY{l+s+s1}{\PYZsq{}}\PY{l+s+s1}{Age}\PY{l+s+s1}{\PYZsq{}}\PY{p}{]}\PY{p}{)}
\end{Verbatim}


    Novamente estou criando um variável para ser usada com a coluna
\textbf{``Embarked''}, pois não quero descartar muitos dados.

    \begin{Verbatim}[commandchars=\\\{\}]
{\color{incolor}In [{\color{incolor}8}]:} \PY{c+c1}{\PYZsh{} criando uma cópia do data frame.}
        \PY{n}{df\PYZus{}emb} \PY{o}{=} \PY{n}{df}\PY{o}{.}\PY{n}{copy}\PY{p}{(}\PY{p}{)}
        
        \PY{c+c1}{\PYZsh{} apagando a coluna \PYZsq{}Age\PYZsq{}.}
        \PY{n}{df\PYZus{}emb}\PY{o}{.}\PY{n}{drop}\PY{p}{(}\PY{p}{[}\PY{l+s+s1}{\PYZsq{}}\PY{l+s+s1}{Age}\PY{l+s+s1}{\PYZsq{}}\PY{p}{]}\PY{p}{,}\PY{n}{axis}\PY{o}{=}\PY{l+m+mi}{1}\PY{p}{,}\PY{n}{inplace}\PY{o}{=}\PY{k+kc}{True}\PY{p}{)}
        
        \PY{c+c1}{\PYZsh{} apagando todas a linhas que possuem elementos nulos.}
        \PY{n}{df\PYZus{}emb}\PY{o}{.}\PY{n}{dropna}\PY{p}{(}\PY{n}{inplace}\PY{o}{=}\PY{k+kc}{True}\PY{p}{)}
\end{Verbatim}


    Para o desensolvimento das questões usarei gráficos, então resolvi criar
funções para o modelos de \textbf{Barra} e \textbf{Pizza}.

    \begin{Verbatim}[commandchars=\\\{\}]
{\color{incolor}In [{\color{incolor}9}]:} \PY{k}{def} \PY{n+nf}{plotting\PYZus{}bars}\PY{p}{(}\PY{n}{data\PYZus{}frame}\PY{p}{:}\PY{p}{(}\PY{n+nb}{dict}\PY{p}{,}\PY{n+nb}{list}\PY{p}{,}\PY{n+nb}{tuple}\PY{p}{)}\PY{p}{,}\PY{n}{title\PYZus{}plot}\PY{p}{:}\PY{n+nb}{str}\PY{p}{,}\PY{n}{x\PYZus{}label}\PY{p}{:}\PY{n+nb}{str}\PY{p}{,}\PY{n}{y\PYZus{}label}\PY{p}{:}\PY{n+nb}{str}\PY{p}{)}\PY{p}{:} 
            
            \PY{l+s+sd}{\PYZdq{}\PYZdq{}\PYZdq{} }
        \PY{l+s+sd}{    Está função irá plotar um gráfico barra de acordo com o dado que for inputado.}
        \PY{l+s+sd}{    Argumentos:}
        \PY{l+s+sd}{        data\PYZus{}frame: lista, dicionário ou tupla inputado.}
        \PY{l+s+sd}{        title\PYZus{}plot: título do gráfico a ser plotado.}
        \PY{l+s+sd}{        x\PYZus{}label: nome da aba do gráfico no eixo x.}
        \PY{l+s+sd}{        y\PYZus{}label: nome da aba do gráfico no eixo y.}
        \PY{l+s+sd}{    Retorna:}
        \PY{l+s+sd}{        Um gráfico em barra com as devidas colunas referentes as variáveis do data\PYZus{}frame.}
        \PY{l+s+sd}{    \PYZdq{}\PYZdq{}\PYZdq{}}
            
            \PY{n}{plt}\PY{o}{.}\PY{n}{xlabel}\PY{p}{(}\PY{n}{x\PYZus{}label}\PY{p}{,}\PY{n}{fontsize}\PY{o}{=}\PY{l+m+mi}{12}\PY{p}{)}\PY{p}{;}
            \PY{n}{plt}\PY{o}{.}\PY{n}{ylabel}\PY{p}{(}\PY{n}{y\PYZus{}label}\PY{p}{,}\PY{n}{fontsize}\PY{o}{=}\PY{l+m+mi}{12}\PY{p}{)}\PY{p}{;}    
            \PY{k}{return} \PY{n}{data\PYZus{}frame}\PY{o}{.}\PY{n}{plot}\PY{p}{(}\PY{n}{kind}\PY{o}{=}\PY{l+s+s2}{\PYZdq{}}\PY{l+s+s2}{bar}\PY{l+s+s2}{\PYZdq{}}\PY{p}{,}\PY{n}{title}\PY{o}{=}\PY{n}{title\PYZus{}plot}\PY{p}{,}\PY{n}{figsize}\PY{o}{=}\PY{p}{(}\PY{l+m+mi}{8}\PY{p}{,}\PY{l+m+mi}{6}\PY{p}{)}\PY{p}{)}\PY{p}{;}  
\end{Verbatim}


    \begin{Verbatim}[commandchars=\\\{\}]
{\color{incolor}In [{\color{incolor}10}]:} \PY{k}{def} \PY{n+nf}{plotting\PYZus{}pie}\PY{p}{(}\PY{n}{data\PYZus{}frame}\PY{p}{:}\PY{p}{(}\PY{n+nb}{list}\PY{p}{,}\PY{n+nb}{dict}\PY{p}{,}\PY{n+nb}{tuple}\PY{p}{)}\PY{p}{,}\PY{n}{title\PYZus{}plot}\PY{p}{:}\PY{n+nb}{str}\PY{p}{,}\PY{n}{explode\PYZus{}distance}\PY{p}{:}\PY{n+nb}{float}\PY{p}{)}\PY{p}{:}
             
             \PY{l+s+sd}{\PYZdq{}\PYZdq{}\PYZdq{}}
         \PY{l+s+sd}{    Está função irá plotar um gráfico pizza de acordo com o dado que for inputado.}
         \PY{l+s+sd}{    Argumentoos:}
         \PY{l+s+sd}{        data\PYZus{}frame: lista,dicionário ou tupla inputado.}
         \PY{l+s+sd}{        title\PYZus{}plot: título do gráfico a ser plotado.}
         \PY{l+s+sd}{        expplode\PYZus{}distance: distância entre as fatias separadas entre si.}
         \PY{l+s+sd}{    Retorna:}
         \PY{l+s+sd}{        Um gráfico em pizza com as variavéis em fatias e suas respectivas porcentagens.}
         \PY{l+s+sd}{    \PYZdq{}\PYZdq{}\PYZdq{}}
             
             \PY{n}{plt}\PY{o}{.}\PY{n}{title}\PY{p}{(}\PY{n}{title\PYZus{}plot}\PY{p}{)}
             \PY{k}{return} \PY{n}{data\PYZus{}frame}\PY{o}{.}\PY{n}{plot}\PY{p}{(}\PY{n}{kind}\PY{o}{=}\PY{l+s+s1}{\PYZsq{}}\PY{l+s+s1}{pie}\PY{l+s+s1}{\PYZsq{}}\PY{p}{,}\PY{n}{figsize}\PY{o}{=}\PY{p}{(}\PY{l+m+mi}{7}\PY{p}{,}\PY{l+m+mi}{7}\PY{p}{)}\PY{p}{,}\PY{n}{explode}\PY{o}{=}\PY{n}{explode\PYZus{}distance}\PY{p}{,}\PY{n}{autopct}\PY{o}{=}\PY{l+s+s1}{\PYZsq{}}\PY{l+s+si}{\PYZpc{}1.0f}\PY{l+s+si}{\PYZpc{}\PYZpc{}}\PY{l+s+s1}{\PYZsq{}}\PY{p}{,}\PY{n}{shadow}\PY{o}{=}\PY{k+kc}{True}\PY{p}{)}\PY{p}{;}
\end{Verbatim}


    \hypertarget{desenvolvimento-das-questuxf5es}{%
\subsection{3 - Desenvolvimento das
Questões}\label{desenvolvimento-das-questuxf5es}}

    \hypertarget{pergunta-1-qual-a-muxe9dia-de-idade-dos-passageiros-que-embarcaram-no-nuxe1vio}{%
\subsubsection{Pergunta 1: Qual a média de idade dos passageiros que
embarcaram no
návio?}\label{pergunta-1-qual-a-muxe9dia-de-idade-dos-passageiros-que-embarcaram-no-nuxe1vio}}

    \begin{Verbatim}[commandchars=\\\{\}]
{\color{incolor}In [{\color{incolor}11}]:} \PY{c+c1}{\PYZsh{} calculando a idade média dos passageiros e usando a função round para arredondar.}
         \PY{n+nb}{print}\PY{p}{(}\PY{l+s+s1}{\PYZsq{}}\PY{l+s+s1}{Idade Média dos Passageiros:}\PY{l+s+s1}{\PYZsq{}}\PY{p}{,}\PY{n+nb}{round}\PY{p}{(}\PY{n}{df\PYZus{}age}\PY{p}{[}\PY{l+s+s1}{\PYZsq{}}\PY{l+s+s1}{Age}\PY{l+s+s1}{\PYZsq{}}\PY{p}{]}\PY{o}{.}\PY{n}{mean}\PY{p}{(}\PY{p}{)}\PY{p}{,}\PY{l+m+mi}{1}\PY{p}{)}\PY{p}{)}
         
         \PY{c+c1}{\PYZsh{} plotando um histograma da coluna \PYZsq{}Age\PYZsq{}.}
         \PY{n}{df\PYZus{}age}\PY{p}{[}\PY{l+s+s1}{\PYZsq{}}\PY{l+s+s1}{Age}\PY{l+s+s1}{\PYZsq{}}\PY{p}{]}\PY{o}{.}\PY{n}{plot}\PY{p}{(}\PY{n}{kind}\PY{o}{=}\PY{l+s+s1}{\PYZsq{}}\PY{l+s+s1}{hist}\PY{l+s+s1}{\PYZsq{}}\PY{p}{,}\PY{n}{title}\PY{o}{=}\PY{l+s+s1}{\PYZsq{}}\PY{l+s+s1}{Média de Idade}\PY{l+s+s1}{\PYZsq{}}\PY{p}{,}\PY{n}{figsize}\PY{o}{=}\PY{p}{(}\PY{l+m+mi}{8}\PY{p}{,}\PY{l+m+mi}{6}\PY{p}{)}\PY{p}{)}\PY{p}{;}
\end{Verbatim}


    \begin{Verbatim}[commandchars=\\\{\}]
Idade Média dos Passageiros: 29.7

    \end{Verbatim}

    \begin{center}
    \adjustimage{max size={0.9\linewidth}{0.9\paperheight}}{output_21_1.png}
    \end{center}
    { \hspace*{\fill} \\}
    
    \hypertarget{pergunta-2---qual-a-muxe9dia-de-idade-por-classe-social-e-genuxearo-do-sobreviventes}{%
\subsubsection{Pergunta 2 - Qual a média de idade por classe social e
genêro do
sobreviventes?}\label{pergunta-2---qual-a-muxe9dia-de-idade-por-classe-social-e-genuxearo-do-sobreviventes}}

    \begin{Verbatim}[commandchars=\\\{\}]
{\color{incolor}In [{\color{incolor}12}]:} \PY{c+c1}{\PYZsh{} filtrando por sobreviventes.}
         \PY{n}{survived\PYZus{}age} \PY{o}{=} \PY{n}{df\PYZus{}age}\PY{o}{.}\PY{n}{query}\PY{p}{(}\PY{l+s+s1}{\PYZsq{}}\PY{l+s+s1}{Survived == }\PY{l+s+s1}{\PYZdq{}}\PY{l+s+s1}{1}\PY{l+s+s1}{\PYZdq{}}\PY{l+s+s1}{\PYZsq{}}\PY{p}{)}
         
         \PY{c+c1}{\PYZsh{} agrupando os sobreviventes por genêro, classe e obtendo a média de idade arrendondada.}
         \PY{n}{survived\PYZus{}group} \PY{o}{=} \PY{n+nb}{round}\PY{p}{(}\PY{n}{survived\PYZus{}age}\PY{o}{.}\PY{n}{groupby}\PY{p}{(}\PY{p}{[}\PY{l+s+s1}{\PYZsq{}}\PY{l+s+s1}{Sex}\PY{l+s+s1}{\PYZsq{}}\PY{p}{,}\PY{l+s+s1}{\PYZsq{}}\PY{l+s+s1}{Pclass}\PY{l+s+s1}{\PYZsq{}}\PY{p}{]}\PY{p}{)}\PY{o}{.}\PY{n}{mean}\PY{p}{(}\PY{p}{)}\PY{p}{[}\PY{l+s+s1}{\PYZsq{}}\PY{l+s+s1}{Age}\PY{l+s+s1}{\PYZsq{}}\PY{p}{]}\PY{p}{,}\PY{l+m+mi}{1}\PY{p}{)}
         \PY{n+nb}{print}\PY{p}{(}\PY{l+s+s1}{\PYZsq{}}\PY{l+s+s1}{Bilhete: }\PY{l+s+se}{\PYZbs{}n}\PY{l+s+s1}{ 1 = Primeira Classe }\PY{l+s+se}{\PYZbs{}n}\PY{l+s+s1}{ 2 = Segunda Classe }\PY{l+s+se}{\PYZbs{}n}\PY{l+s+s1}{ 3 = Terceira Classe}\PY{l+s+s1}{\PYZsq{}}\PY{p}{)}
         \PY{n+nb}{print}\PY{p}{(}\PY{l+s+s2}{\PYZdq{}}\PY{l+s+se}{\PYZbs{}n}\PY{l+s+s2}{\PYZdq{}}\PY{p}{,} \PY{n}{survived\PYZus{}group}\PY{p}{,}\PY{l+s+s2}{\PYZdq{}}\PY{l+s+se}{\PYZbs{}n}\PY{l+s+s2}{\PYZdq{}}\PY{p}{)}
         
         \PY{c+c1}{\PYZsh{} plotando um gráfico pizza.}
         \PY{n}{plotting\PYZus{}pie}\PY{p}{(}\PY{n}{survived\PYZus{}group}\PY{p}{,}\PY{l+s+s1}{\PYZsq{}}\PY{l+s+s1}{Idade Média por Classe Social e Genêro}\PY{l+s+s1}{\PYZsq{}}\PY{p}{,}\PY{p}{(}\PY{l+m+mf}{0.05}\PY{p}{,}\PY{l+m+mf}{0.05}\PY{p}{,}\PY{l+m+mf}{0.05}\PY{p}{,}\PY{l+m+mf}{0.05}\PY{p}{,}\PY{l+m+mf}{0.05}\PY{p}{,}\PY{l+m+mf}{0.05}\PY{p}{)}\PY{p}{)}
\end{Verbatim}


    \begin{Verbatim}[commandchars=\\\{\}]
Bilhete: 
 1 = Primeira Classe 
 2 = Segunda Classe 
 3 = Terceira Classe

 Sex     Pclass
female  1         34.9
        2         28.1
        3         19.3
male    1         36.2
        2         15.9
        3         22.3
Name: Age, dtype: float64 


    \end{Verbatim}

\begin{Verbatim}[commandchars=\\\{\}]
{\color{outcolor}Out[{\color{outcolor}12}]:} <matplotlib.axes.\_subplots.AxesSubplot at 0x4c9a090b00>
\end{Verbatim}
            
    \begin{center}
    \adjustimage{max size={0.9\linewidth}{0.9\paperheight}}{output_23_2.png}
    \end{center}
    { \hspace*{\fill} \\}
    
    \hypertarget{pergunta-3-qual-uxe9-a-classe-economica-predominante}{%
\subsubsection{Pergunta 3: Qual é a classe economica
predominante?}\label{pergunta-3-qual-uxe9-a-classe-economica-predominante}}

    \begin{Verbatim}[commandchars=\\\{\}]
{\color{incolor}In [{\color{incolor}13}]:} \PY{c+c1}{\PYZsh{} contando os valores da coluna classe social.}
         \PY{n+nb}{print}\PY{p}{(}\PY{l+s+s1}{\PYZsq{}}\PY{l+s+s1}{Bilhete: }\PY{l+s+se}{\PYZbs{}n}\PY{l+s+s1}{ 1 = Primeira Classe }\PY{l+s+se}{\PYZbs{}n}\PY{l+s+s1}{ 2 = Segunda Classe }\PY{l+s+se}{\PYZbs{}n}\PY{l+s+s1}{ 3 = Terceira Classe }\PY{l+s+se}{\PYZbs{}n}\PY{l+s+s1}{\PYZsq{}}\PY{p}{)}
         \PY{n+nb}{print}\PY{p}{(}\PY{n}{df}\PY{p}{[}\PY{l+s+s1}{\PYZsq{}}\PY{l+s+s1}{Pclass}\PY{l+s+s1}{\PYZsq{}}\PY{p}{]}\PY{o}{.}\PY{n}{value\PYZus{}counts}\PY{p}{(}\PY{p}{)}\PY{p}{)}
         
         \PY{c+c1}{\PYZsh{} Usando a função para plotar um gráfico barra.}
         \PY{n}{plotting\PYZus{}bars}\PY{p}{(}\PY{n}{df}\PY{p}{[}\PY{l+s+s1}{\PYZsq{}}\PY{l+s+s1}{Pclass}\PY{l+s+s1}{\PYZsq{}}\PY{p}{]}\PY{o}{.}\PY{n}{value\PYZus{}counts}\PY{p}{(}\PY{p}{)}\PY{p}{,}\PY{l+s+s1}{\PYZsq{}}\PY{l+s+s1}{Bilhete de Passagem}\PY{l+s+s1}{\PYZsq{}}\PY{p}{,}\PY{l+s+s1}{\PYZsq{}}\PY{l+s+s1}{Tipo de Bilhete}\PY{l+s+s1}{\PYZsq{}}\PY{p}{,}\PY{l+s+s1}{\PYZsq{}}\PY{l+s+s1}{Quantidade}\PY{l+s+s1}{\PYZsq{}}\PY{p}{)}
\end{Verbatim}


    \begin{Verbatim}[commandchars=\\\{\}]
Bilhete: 
 1 = Primeira Classe 
 2 = Segunda Classe 
 3 = Terceira Classe 

3    491
1    216
2    184
Name: Pclass, dtype: int64

    \end{Verbatim}

\begin{Verbatim}[commandchars=\\\{\}]
{\color{outcolor}Out[{\color{outcolor}13}]:} <matplotlib.axes.\_subplots.AxesSubplot at 0x4c9a4e7320>
\end{Verbatim}
            
    \begin{center}
    \adjustimage{max size={0.9\linewidth}{0.9\paperheight}}{output_25_2.png}
    \end{center}
    { \hspace*{\fill} \\}
    
    \hypertarget{pergunta-4-cidades-de-embarque-por-classe-social.}{%
\subsubsection{Pergunta 4: Cidades de Embarque por Classe
Social.}\label{pergunta-4-cidades-de-embarque-por-classe-social.}}

    \begin{itemize}
\tightlist
\item
  \textbf{Aspecto geral das localidades de embarque.}
\end{itemize}

    \begin{Verbatim}[commandchars=\\\{\}]
{\color{incolor}In [{\color{incolor}14}]:} \PY{c+c1}{\PYZsh{} contando os elementos da coluna \PYZsq{}Embarked\PYZsq{}.}
         \PY{n+nb}{print}\PY{p}{(}\PY{l+s+s1}{\PYZsq{}}\PY{l+s+s1}{Cidades: }\PY{l+s+se}{\PYZbs{}n}\PY{l+s+s1}{ C = Cherbourg }\PY{l+s+se}{\PYZbs{}n}\PY{l+s+s1}{ S = Southampton }\PY{l+s+se}{\PYZbs{}n}\PY{l+s+s1}{ Q = Queenstown }\PY{l+s+se}{\PYZbs{}n}\PY{l+s+s1}{\PYZsq{}}\PY{p}{)}
         \PY{n+nb}{print}\PY{p}{(}\PY{n}{df\PYZus{}emb}\PY{p}{[}\PY{l+s+s1}{\PYZsq{}}\PY{l+s+s1}{Embarked}\PY{l+s+s1}{\PYZsq{}}\PY{p}{]}\PY{o}{.}\PY{n}{value\PYZus{}counts}\PY{p}{(}\PY{p}{)}\PY{p}{)}
         
         \PY{c+c1}{\PYZsh{}plotando o gráfico através da função}
         \PY{n}{plotting\PYZus{}bars}\PY{p}{(}\PY{n}{df\PYZus{}emb}\PY{p}{[}\PY{l+s+s1}{\PYZsq{}}\PY{l+s+s1}{Embarked}\PY{l+s+s1}{\PYZsq{}}\PY{p}{]}\PY{o}{.}\PY{n}{value\PYZus{}counts}\PY{p}{(}\PY{p}{)}\PY{p}{,}\PY{l+s+s1}{\PYZsq{}}\PY{l+s+s1}{Cidade de Embarque}\PY{l+s+s1}{\PYZsq{}}\PY{p}{,}\PY{l+s+s1}{\PYZsq{}}\PY{l+s+s1}{Cidades}\PY{l+s+s1}{\PYZsq{}}\PY{p}{,}\PY{l+s+s1}{\PYZsq{}}\PY{l+s+s1}{Passageiros}\PY{l+s+s1}{\PYZsq{}}\PY{p}{)}
\end{Verbatim}


    \begin{Verbatim}[commandchars=\\\{\}]
Cidades: 
 C = Cherbourg 
 S = Southampton 
 Q = Queenstown 

S    644
C    168
Q     77
Name: Embarked, dtype: int64

    \end{Verbatim}

\begin{Verbatim}[commandchars=\\\{\}]
{\color{outcolor}Out[{\color{outcolor}14}]:} <matplotlib.axes.\_subplots.AxesSubplot at 0x4c9a5528d0>
\end{Verbatim}
            
    \begin{center}
    \adjustimage{max size={0.9\linewidth}{0.9\paperheight}}{output_28_2.png}
    \end{center}
    { \hspace*{\fill} \\}
    
    \begin{itemize}
\tightlist
\item
  \textbf{3.1 - CHERBOURG}
\end{itemize}

    \begin{Verbatim}[commandchars=\\\{\}]
{\color{incolor}In [{\color{incolor}15}]:} \PY{c+c1}{\PYZsh{} filtrando os passageiros que embarcaram somente em cherbourg.}
         \PY{n}{cherbourg} \PY{o}{=} \PY{n}{df\PYZus{}emb}\PY{o}{.}\PY{n}{query}\PY{p}{(}\PY{l+s+s1}{\PYZsq{}}\PY{l+s+s1}{Embarked == }\PY{l+s+s1}{\PYZdq{}}\PY{l+s+s1}{C}\PY{l+s+s1}{\PYZdq{}}\PY{l+s+s1}{\PYZsq{}}\PY{p}{)}
         
         \PY{c+c1}{\PYZsh{} comparando os passageiros de cherbourg por classe social.}
         \PY{n+nb}{print}\PY{p}{(}\PY{l+s+s1}{\PYZsq{}}\PY{l+s+s1}{Bilhete: }\PY{l+s+se}{\PYZbs{}n}\PY{l+s+s1}{ 1 = Primeira Classe }\PY{l+s+se}{\PYZbs{}n}\PY{l+s+s1}{ 2 = Segunda Classe }\PY{l+s+se}{\PYZbs{}n}\PY{l+s+s1}{ 3 = Terceira Classe }\PY{l+s+se}{\PYZbs{}n}\PY{l+s+s1}{\PYZsq{}}\PY{p}{)}
         \PY{n+nb}{print}\PY{p}{(}\PY{n}{cherbourg}\PY{p}{[}\PY{l+s+s1}{\PYZsq{}}\PY{l+s+s1}{Pclass}\PY{l+s+s1}{\PYZsq{}}\PY{p}{]}\PY{o}{.}\PY{n}{value\PYZus{}counts}\PY{p}{(}\PY{p}{)}\PY{p}{)}
         
         \PY{c+c1}{\PYZsh{} plotando um gráfico barra para a contagem de passageiros por classe social.}
         \PY{n}{plotting\PYZus{}bars}\PY{p}{(}\PY{n}{cherbourg}\PY{p}{[}\PY{l+s+s1}{\PYZsq{}}\PY{l+s+s1}{Pclass}\PY{l+s+s1}{\PYZsq{}}\PY{p}{]}\PY{o}{.}\PY{n}{value\PYZus{}counts}\PY{p}{(}\PY{p}{)}\PY{p}{,}\PY{l+s+s1}{\PYZsq{}}\PY{l+s+s1}{Bilhete em Cherbourg}\PY{l+s+s1}{\PYZsq{}}\PY{p}{,}\PY{l+s+s1}{\PYZsq{}}\PY{l+s+s1}{Classe do Bilhete}\PY{l+s+s1}{\PYZsq{}}\PY{p}{,}\PY{l+s+s1}{\PYZsq{}}\PY{l+s+s1}{Quantidade}\PY{l+s+s1}{\PYZsq{}}\PY{p}{)}
\end{Verbatim}


    \begin{Verbatim}[commandchars=\\\{\}]
Bilhete: 
 1 = Primeira Classe 
 2 = Segunda Classe 
 3 = Terceira Classe 

1    85
3    66
2    17
Name: Pclass, dtype: int64

    \end{Verbatim}

\begin{Verbatim}[commandchars=\\\{\}]
{\color{outcolor}Out[{\color{outcolor}15}]:} <matplotlib.axes.\_subplots.AxesSubplot at 0x4c9a5a70b8>
\end{Verbatim}
            
    \begin{center}
    \adjustimage{max size={0.9\linewidth}{0.9\paperheight}}{output_30_2.png}
    \end{center}
    { \hspace*{\fill} \\}
    
    \begin{itemize}
\tightlist
\item
  \textbf{3.2 - QUEENSTOWN}
\end{itemize}

    \begin{Verbatim}[commandchars=\\\{\}]
{\color{incolor}In [{\color{incolor}16}]:} \PY{c+c1}{\PYZsh{} filtrando os passageiros que embarcaram somente em queenstown.}
         \PY{n}{queenstown} \PY{o}{=} \PY{n}{df\PYZus{}emb}\PY{o}{.}\PY{n}{query}\PY{p}{(}\PY{l+s+s1}{\PYZsq{}}\PY{l+s+s1}{Embarked == }\PY{l+s+s1}{\PYZdq{}}\PY{l+s+s1}{Q}\PY{l+s+s1}{\PYZdq{}}\PY{l+s+s1}{\PYZsq{}}\PY{p}{)}
         
         \PY{n+nb}{print}\PY{p}{(}\PY{l+s+s1}{\PYZsq{}}\PY{l+s+s1}{Bilhete: }\PY{l+s+se}{\PYZbs{}n}\PY{l+s+s1}{ 1 = Primeira Classe }\PY{l+s+se}{\PYZbs{}n}\PY{l+s+s1}{ 2 = Segunda Classe }\PY{l+s+se}{\PYZbs{}n}\PY{l+s+s1}{ 3 = Terceira Classe }\PY{l+s+se}{\PYZbs{}n}\PY{l+s+s1}{\PYZsq{}}\PY{p}{)}
         \PY{n+nb}{print}\PY{p}{(}\PY{n}{queenstown}\PY{p}{[}\PY{l+s+s1}{\PYZsq{}}\PY{l+s+s1}{Pclass}\PY{l+s+s1}{\PYZsq{}}\PY{p}{]}\PY{o}{.}\PY{n}{value\PYZus{}counts}\PY{p}{(}\PY{p}{)}\PY{p}{)}
         
         \PY{c+c1}{\PYZsh{} plotando um gráfico barra para mostrar a quantidade por grupo social}
         \PY{n}{plotting\PYZus{}bars}\PY{p}{(}\PY{n}{queenstown}\PY{p}{[}\PY{l+s+s1}{\PYZsq{}}\PY{l+s+s1}{Pclass}\PY{l+s+s1}{\PYZsq{}}\PY{p}{]}\PY{o}{.}\PY{n}{value\PYZus{}counts}\PY{p}{(}\PY{p}{)}\PY{p}{,}\PY{l+s+s1}{\PYZsq{}}\PY{l+s+s1}{Bilhete em Queenstown}\PY{l+s+s1}{\PYZsq{}}\PY{p}{,}\PY{l+s+s1}{\PYZsq{}}\PY{l+s+s1}{Classe do Bilhete}\PY{l+s+s1}{\PYZsq{}}\PY{p}{,}\PY{l+s+s1}{\PYZsq{}}\PY{l+s+s1}{Quantidade}\PY{l+s+s1}{\PYZsq{}}\PY{p}{)}
\end{Verbatim}


    \begin{Verbatim}[commandchars=\\\{\}]
Bilhete: 
 1 = Primeira Classe 
 2 = Segunda Classe 
 3 = Terceira Classe 

3    72
2     3
1     2
Name: Pclass, dtype: int64

    \end{Verbatim}

\begin{Verbatim}[commandchars=\\\{\}]
{\color{outcolor}Out[{\color{outcolor}16}]:} <matplotlib.axes.\_subplots.AxesSubplot at 0x4c99f2aa20>
\end{Verbatim}
            
    \begin{center}
    \adjustimage{max size={0.9\linewidth}{0.9\paperheight}}{output_32_2.png}
    \end{center}
    { \hspace*{\fill} \\}
    
    \begin{itemize}
\tightlist
\item
  \textbf{3.3 - SOUTHAMPTON}
\end{itemize}

    \begin{Verbatim}[commandchars=\\\{\}]
{\color{incolor}In [{\color{incolor}17}]:} \PY{c+c1}{\PYZsh{} filtrando os passageiros que somente embarcaram em southampton.}
         \PY{n}{southampton} \PY{o}{=} \PY{n}{df\PYZus{}emb}\PY{o}{.}\PY{n}{query}\PY{p}{(}\PY{l+s+s1}{\PYZsq{}}\PY{l+s+s1}{Embarked == }\PY{l+s+s1}{\PYZdq{}}\PY{l+s+s1}{S}\PY{l+s+s1}{\PYZdq{}}\PY{l+s+s1}{\PYZsq{}}\PY{p}{)}
         
         \PY{c+c1}{\PYZsh{} contando os passageiros de southampton por classe social.}
         \PY{n+nb}{print}\PY{p}{(}\PY{l+s+s1}{\PYZsq{}}\PY{l+s+s1}{Bilhete: }\PY{l+s+se}{\PYZbs{}n}\PY{l+s+s1}{ 1 = Primeira Classe }\PY{l+s+se}{\PYZbs{}n}\PY{l+s+s1}{ 2 = Segunda Classe }\PY{l+s+se}{\PYZbs{}n}\PY{l+s+s1}{ 3 = Terceira Classe }\PY{l+s+se}{\PYZbs{}n}\PY{l+s+s1}{\PYZsq{}}\PY{p}{)}
         \PY{n+nb}{print}\PY{p}{(}\PY{n}{southampton}\PY{p}{[}\PY{l+s+s1}{\PYZsq{}}\PY{l+s+s1}{Pclass}\PY{l+s+s1}{\PYZsq{}}\PY{p}{]}\PY{o}{.}\PY{n}{value\PYZus{}counts}\PY{p}{(}\PY{p}{)}\PY{p}{)}
         
         \PY{c+c1}{\PYZsh{} plotando os passageiros em gráfico de barra, mostrados por grupo social.}
         \PY{n}{plotting\PYZus{}bars}\PY{p}{(}\PY{n}{southampton}\PY{p}{[}\PY{l+s+s1}{\PYZsq{}}\PY{l+s+s1}{Pclass}\PY{l+s+s1}{\PYZsq{}}\PY{p}{]}\PY{o}{.}\PY{n}{value\PYZus{}counts}\PY{p}{(}\PY{p}{)}\PY{p}{,}\PY{l+s+s1}{\PYZsq{}}\PY{l+s+s1}{Bilhete em Southampton}\PY{l+s+s1}{\PYZsq{}}\PY{p}{,}\PY{l+s+s1}{\PYZsq{}}\PY{l+s+s1}{Classe do Bilhete}\PY{l+s+s1}{\PYZsq{}}\PY{p}{,}\PY{l+s+s1}{\PYZsq{}}\PY{l+s+s1}{Quantidade}\PY{l+s+s1}{\PYZsq{}}\PY{p}{)}
\end{Verbatim}


    \begin{Verbatim}[commandchars=\\\{\}]
Bilhete: 
 1 = Primeira Classe 
 2 = Segunda Classe 
 3 = Terceira Classe 

3    353
2    164
1    127
Name: Pclass, dtype: int64

    \end{Verbatim}

\begin{Verbatim}[commandchars=\\\{\}]
{\color{outcolor}Out[{\color{outcolor}17}]:} <matplotlib.axes.\_subplots.AxesSubplot at 0x4c9a66e940>
\end{Verbatim}
            
    \begin{center}
    \adjustimage{max size={0.9\linewidth}{0.9\paperheight}}{output_34_2.png}
    \end{center}
    { \hspace*{\fill} \\}
    
    \hypertarget{pergunta-5---entre-os-sobreviventes-qual-o-sexo-predominou}{%
\subsubsection{Pergunta 5 - Entre os sobreviventes, qual o sexo
predominou?}\label{pergunta-5---entre-os-sobreviventes-qual-o-sexo-predominou}}

    \begin{Verbatim}[commandchars=\\\{\}]
{\color{incolor}In [{\color{incolor}18}]:} \PY{c+c1}{\PYZsh{} filtrando os sobreviventes.}
         \PY{n}{survived} \PY{o}{=} \PY{n}{df}\PY{o}{.}\PY{n}{query}\PY{p}{(}\PY{l+s+s1}{\PYZsq{}}\PY{l+s+s1}{Survived == }\PY{l+s+s1}{\PYZdq{}}\PY{l+s+s1}{1}\PY{l+s+s1}{\PYZdq{}}\PY{l+s+s1}{\PYZsq{}}\PY{p}{)}
         
         \PY{c+c1}{\PYZsh{} contandos o numero de sobreviventes por sexo.}
         \PY{n+nb}{print}\PY{p}{(}\PY{l+s+s1}{\PYZsq{}}\PY{l+s+s1}{Gênero:}\PY{l+s+s1}{\PYZsq{}}\PY{p}{)}
         \PY{n+nb}{print}\PY{p}{(}\PY{n}{survived}\PY{p}{[}\PY{l+s+s1}{\PYZsq{}}\PY{l+s+s1}{Sex}\PY{l+s+s1}{\PYZsq{}}\PY{p}{]}\PY{o}{.}\PY{n}{value\PYZus{}counts}\PY{p}{(}\PY{p}{)}\PY{p}{)}
         
         \PY{c+c1}{\PYZsh{} plotando um gráfico pizza.}
         \PY{n}{plotting\PYZus{}pie}\PY{p}{(}\PY{n}{survived}\PY{p}{[}\PY{l+s+s1}{\PYZsq{}}\PY{l+s+s1}{Sex}\PY{l+s+s1}{\PYZsq{}}\PY{p}{]}\PY{o}{.}\PY{n}{value\PYZus{}counts}\PY{p}{(}\PY{p}{)}\PY{p}{,}\PY{l+s+s1}{\PYZsq{}}\PY{l+s+s1}{Gênero dos Sobreviventes}\PY{l+s+s1}{\PYZsq{}}\PY{p}{,}\PY{p}{(}\PY{l+m+mi}{0}\PY{p}{,}\PY{l+m+mf}{0.1}\PY{p}{)}\PY{p}{)}
\end{Verbatim}


    \begin{Verbatim}[commandchars=\\\{\}]
Gênero:
female    233
male      109
Name: Sex, dtype: int64

    \end{Verbatim}

\begin{Verbatim}[commandchars=\\\{\}]
{\color{outcolor}Out[{\color{outcolor}18}]:} <matplotlib.axes.\_subplots.AxesSubplot at 0x4c9a6b0518>
\end{Verbatim}
            
    \begin{center}
    \adjustimage{max size={0.9\linewidth}{0.9\paperheight}}{output_36_2.png}
    \end{center}
    { \hspace*{\fill} \\}
    
    \hypertarget{conclusuxe3o}{%
\subsection{4 - Conclusão}\label{conclusuxe3o}}

\begin{itemize}
\tightlist
\item
  \emph{A média de idade dos passageiros é de 29 anos, grande maioria
  tinha idade entre 20 à 30 anos.}
\item
  \emph{A média de idade dos sobreviventes por classe social entre
  mulheres: 1ª classe é de 34 anos, 2ª classe 28 anos, 3ª classe 19
  anos.}
\item
  \emph{A média de idade dos sobreviventes por classe social entre
  homens são: 1ª classe é de 36 anos, 2ª classe 15 anos, 3ª classe 22
  anos.}
\item
  \emph{Os bilhetes mais vendidos foram o de 3ª classe, esta quantidade
  de passagens mostrou que a predominância social era de renda mais
  baixa.}
\item
  \emph{A cidade que mais pessoas embarcaram foi Southampton.}
\item
  \emph{Em Southamton a predominância foi de passagens 3ª classe,
  conclcuindo assim que esses passageiros pertenciam a classe baixa.}
\item
  \emph{Em Cherbourg, observou que a ocorrência maior são para
  passageiros de bilheres 1ª classe, relacionando assim a classe com
  poder aquisitivo maior.}
\item
  \emph{Em Queenstown, passagens de 3ª classe teve maior frequência.}
\item
  \emph{Entre os sobreviventes foi observado que cerca de 68\% eram
  mulheres e 32\% eram homens.}
\end{itemize}

\hypertarget{limitauxe7uxf5es}{%
\subsubsection{Limitações}\label{limitauxe7uxf5es}}

Foi identificado que existiam colunas que faltavam dados sobre alguns
passageiros, essa colunas são: ``Age'',``Cabin'' e ``Embarked''.

Medida realizada: * Para a remoção dos dados nulos da coluna Age foi
criado uma variável espeficica para evitar a perda de dados não
desejados. * Aplicação parecida com a da coluna ``Age'' foi feita em
``Embarked'', assim evitando que dados fossem desperdiçados. *
``Cabin'', não há referência das posições das cabines, seria
interessante saber a taxa de sobrevivência por local do barco.

Coluna excluidas:

\begin{itemize}
\tightlist
\item
  A coluna ``PassengerId'' foi excluida pois o próprio index do data
  frame já orienta quanto ao número de linhas.
\item
  ``Name'' foi retirada pois os nomes não são relevante nesses estudo.
\item
  ``Ticket'', se refere somente ao numero do bilhete, não produz insight
  neste estudo.
\item
  ``Cabin'', não há referência das posições das cabines, seria
  interessante saber a taxa de sobrevivência por local do barco.
\end{itemize}

Também foram modificados propriedade de algumas colunas, como ``Age'' e
``Fare'',

Algumas fontes foram úteis para desenvolvimento de parte dos código, são
estas: *
\href{https://matplotlib.org/api/_as_gen/matplotlib.pyplot.pie.html\#matplotlib.pyplot.pie}{Matplotlib.pyplot}
* \href{https://docs.scipy.org/doc/numpy/user/basics.types.html}{Numpy}
*
\href{https://pandas.pydata.org/pandas-docs/stable/visualization.html}{Pandas}


    % Add a bibliography block to the postdoc
    
    
    
    \end{document}
